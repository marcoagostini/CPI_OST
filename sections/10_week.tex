%You know when and why to prefer standard algorithms over hand-written loops 
%You can name the most important algorithms in the STL 
%You can explain certain pittfalls when using STL algorithms 
%You can explain the signature of a standard algorithm 
%You can write programs that correctly use standard algorithms


\section{Class Templates}

\begin{itemize}
  \itemsep 0em 
  \item In addition to functions also class types can habe template parameters
  \item Since C++17, similar to function templates, the compiler might deduce the template arguments
  \item Class templates deliver types with compile time parameters
  \item Member function which never are used are also never compiled
  \item Compile-time polymorphism
  \item Class templates can be specialized
\end{itemize}

\textbf{Rules}
\begin{itemize}
  \itemsep 0em 
  \item Define class templates completely in header files!
  \item Member functions of class templates
  	\SubItem{Either in class template directly}
  	\SubItem{Or as inline function templates in the same header file}
  \item When using language elements depending directly or indirectly on a template parameter, you must specify typename when it is naming a type.
  \item static member variables of a template class can be defined in header without violating ODR, even if included in several compilation units.
\end{itemize}

\begin{lstlisting}
template <TemplateParameters>
class TemplateName { /*...*/ };

template <typename T>
class Stack { /*...*/ };
\end{lstlisting}

\subsection{Template Argument Deduction (C++17)}
Similar to function templates, the compiler might deduce the template arguments. This is a compile-time polmorphism.
\begin{lstlisting}[language=C++]
std::vector newValues{1, 2, 3}; // The compiler can deduce the type 
std::vector<int> emptyValues{}; 
\end{lstlisting}

\subsection{Type Alias \& Dependent Names}
\begin{itemize}
  \itemsep -0.5em 
  \item It is common for template definitions to define type aliases in order to ease their use.
  \item Within the template definition you might use names that are directly or indirectly depending on the template parameter.
  \item Dependent Name: Compiler geht standardmässig davon aus, dass es sich im eine Variable, oder eine Funktion handelt. Wenn es ein Typ ist (wie size\_type), muss das keyword typename verwendet werden.
\end{itemize}

Example
\begin{lstlisting}[language=C++]
template <typename T> // Class template with one typename par
class Sack {
	using SackType = std::vector<T>;
	using size_type = typename SackType::size_type; // dependent name
	SackType theSack{};
	
public:
	bool empty() const {
		return theSack.empty();
	}
	size_type size() const {
		return theSack.size();
	}
	void putInto(T const & item) {
		theSack.push_back(item);
	}
	T getOut(); // member forward declaration
};
\end{lstlisting}
Define the function outside of the template class definition.
\begin{lstlisting}[language=C++]
template <typename T> 
inline T Sack<T>::getOut() { // implementation outside of class
	if (empty()) {
		throw std::logic_error{"Empty Sack"}; 
	} 
	auto index = static_cast<size_type>(rand() % size()); 
	T retval{theSack.at(index)};
	theSack.erase(theSack.begin() + index);
	return retval;
}
\end{lstlisting}

\subsection{Inheritance}
Rule: Always use this->variable (or className::) to refer to inherited members in a template class.

\pagebreak
