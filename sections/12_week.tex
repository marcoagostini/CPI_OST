\section{Inheritance}
Inheritance is always then used, when specific components want to be reused and extended. Inheritance can be bad because it creates a very strong dependency.

\begin{itemize}
  \itemsep -0.5em 
  \item Inheritance is default public (Classes). For structs the inheritance is private.
  \item The constructors are not inherited implicitly we have to specify that.
  \item The parent is always constructed first and after that the children.
  \item Assining or send parameters per value from an inherited class to the base class result in \textbf{Object Slicing}.
\end{itemize}

\begin{lstlisting}[language=C++]
class MyClass : Base {}; // implicit private 
struct MyStruct : Base {}; // implicit public 

class MyClass : public Base {
	public:
		 using Base::Base; // inherit constructor
};
\end{lstlisting}

\subsection{Initialising Multiple Base Classes}
 Base constructors can be explicitly called in the member initializer list. You should put base class constructor class before the initialization of members. The compiler enforces this rule, even though you can put the list of initializers in wrong order.
 
 \begin{lstlisting}[language=C++]
class DerivedWithCtor : public Base1, public Base2 {
	int mvar; 
public:
	// calls base1, base2, mvar
	DerivedWithCtor(int i, int j) : Base1{i}, Base2{}, mvar{j} {}
};
\end{lstlisting}

\subsection{Dynamic Polymorphism}
\begin{itemize}
  \itemsep -0.5em 
  \item Operator and function overloading and templates allow polymorphic behaviour at compile time
  \item Dynamic polymorphism needs object references or (smart) pointers to work
  	\SubItem{Syntax overhead}
  	\SubItem{The base class must have a good abstratction}
  	\SubItem{Copying carries the danger of slicing (partial copying)}
\end{itemize}

\subsubsection{Shadowing Member Functions}
\begin{itemize}
  \itemsep -0.5em 
  \item if a function is reimplemented in a derived class, it shadows its counterpart in the base class
  \item However, if accessed through a declared bases object, the shadowing function is ignored
\end{itemize}

\begin{lstlisting}[language=C++]
struct Base {
	// shadowed function 
	void sayHello() const {
		"Im Base\n"
	}
}
struct Derived : Base {
	// shadowing function
	void sayHello() const {
		std::cout << "hi, im derived\n";
	}
};
void greet(Base const & base) {
	base.sayHello(); 
}
in main() {
	Derived derived{};
	greet(derived); // Hi, im Base (static call)
}
\end{lstlisting}

\subsection{Virtual Member Functions}
\begin{itemize}
  \itemsep -0.5em 
  \item To achieve dynamic polymorphism "virtual" member functions are required
  \item "Virtual" member functions are bound dynamically.
  \item The virtual keyword ist automatically inherited and does not have to be restated at childs.
  \item Its possible to state overriding functions with "override"
  \item To override a virtual function the signatures habe to be the same!
\end{itemize}

\begin{lstlisting}[language=C++]
struct Base {
  virtual void sayHello() const {
    std::cout << "Hi, I'm Base\n";
  }
};

struct Derived : Base {
  void sayHello() const { // virtual is automatically inhertited
    std::cout << "Hi, I'm Derived\n";
  }
};
void greet(Base const & base) {
  base.sayHello();
}

int main() {
  Derived derived{};
  greet(derived); // Hi, I'm Derived (dynamic call)
}
\end{lstlisting}

\subsection{Calling Virtual Member Functions}
\begin{itemize}
  \itemsep -0.5em 
  \item Value Object
  	\SubItem{Class type determines function, regardless of virtual}
  \item	Reference
  	\SubItem{Virtual member of derived class called through base class reference}
  \item Smart Pointer
  	\SubItem{Virtual member of derived class called through smart pointer to base class}
  \item Dump Poiner (rarely used)
  	\SubItem{Virtual member of derived class called through base class pointer}
\end{itemize}

\begin{lstlisting}[language=C++]
void greet(Base base) {
	base.sayHello(); // Value: always calls base
}

void greet(Base & base) {
	base.sayHello(); // Reference: dynamic binding
}

void greet(std::unique_ptr<Base> base) {
	base.sayHello(); // dynamic binding
}

void greet(Base const * base) {
	base->sayHello(); // dynamic binding
}
\end{lstlisting}

\subsection{Abstract Base Classes}
\begin{itemize}
  \itemsep -0.5em 
  \item There are no Interfaces in C++
  \item A pure virtual member function makes a class abstract
  \item To mark a virtual member function as pure virtual it has zero assigned after its signature
  \item Abstract classes cannot be instantiated (like in Java)
\end{itemize}

\begin{lstlisting}[language=C++]
struct abstractBase {
	virtual void doitnow() = 0;
}
\end{lstlisting}

\subsection{Destructors}
\begin{itemize}
  \itemsep -0.5em 
  \item Classes with virtual members require a virtual Destructor
  \item Otherwise when allocated on the heap with make\_unique and assigned to a unique\_ptr only the destructor of Base is called
\end{itemize}

\begin{lstlisting}[language=C++]
struct Fuel {
  virtual void burn() = 0;
  virtual ~Fuel() { std::cout << "put into trash\n" }
};

struct Plutonium : Fuel {
  void burn() { std::cout << "split core\n"; }
  ~Plutonium() { std::cout << "store many years\n"; }
};

int main() {
  std::unique_ptr<Fuel> surprise = std::make_unique<Plutonium>(); // both called
}
\end{lstlisting}

\subsection{Object Slicing}
The object slicing problem can be solved if we set the copy operations as delted.
\begin{lstlisting}[language=C++]
struct Base {
	Base & operator=(Base const & other) = deleted;
	Book(Book const & other) = deleted;
}
\end{lstlisting}


