% Goals
% You can explain the difference between Java Generics and C++ Templates
% You can implement simple generic functions
% You can implement variadic generic functions
% You can describe the type concepts required by a function template

\section{Function Templates}
\begin{itemize}
  \itemsep -0.5em 
  \item Can be compared as a Generic in Java. The keyword "template" is used to declare a template.
  \item The template parameter list contains one or more templates parameters.
  \item C++ uses duck-typing. So every type can be used as argument as long as it supports the used operations.
  \item Function templates are normally defined and implemented in a header file.
  \item Tempalate functions are implicitly inline
  \item We can write generics with templates.
\end{itemize}

The compiler resolves the function template and figures out the template arguments. 
\begin{lstlisting}[language=C++]
template <Template-Parameter-List>
FunctionDefinition
\end{lstlisting}

\begin{lstlisting}[language=C++]
// file min.h
template <typename T>
T min(T left, T right) {
	return left < right ? left : right;
}
// file smaller.cpp
#include "min.h"
#include <iostream>
int main() {
	int first;
	int second;
	if (std::cin >> first >> second) {
		auto const smaller = min(first, second); std::cout << "Smaller of " << first << " and " << second << " is: " << smaller << '\n';
	}
}
\end{lstlisting}

\textbf{Template Argument Deduction}


\subsection{Variadic Templates}
\begin{itemize}
  \itemsep -0.5em 
  \item For function templates with an arbitrary number of parameters
  \item Needs at least one pack parameter
  \item Pack Expansion: For each argument in that pack an instance of the pattern is created
  \item In an instance of the pattern the parameter pack name is replaced by an argument of the pack
  \item Needs a base case for the recursion (after the last parameter is done, it would call the function without a parameter, which is invalid) $\rightarrow$ Base case must be written before the template function.
\end{itemize}
\begin{lstlisting}[language=C++]
#include <iostream>
#include <string>

// Base Case 
void printAll(std::ostream & out) {
}

template<typename First, typename...Types>
void printAll(std::ostream & out, First const & first, Types const &...rest) {
  out << first;
  if (sizeof...(Types)) {
    out << ", ";
  }
  printAll(out, rest...);
}

int main() {
	int i{42}; double d{1.25}; std::string book{"Lucid C++"};
	printAll(std::cout, i, d, book);
}

\end{lstlisting}
