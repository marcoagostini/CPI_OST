\section{Introduction}
\subsection{Why C++?}
\begin{itemize}
  \itemsep0em 
  \item Work al almost all platforms from a micro controller to the main frame
  \item Multi-paradigm language with zero-cost abstraction
  \item High-level abstraction facilities
  \item The concepts from C++ can mostly be applied to any other programming language
\end{itemize}

\subsection{Undefined Behaviour}
The undefined behaviour is defined in the C++ standard (funny, isn't it?). C++ has no garbage collector. If in C++ something is written wrong and the compiler doesn't detect it: undefined behaviour can occur.

\subsection{C++ Compilation Process}
\textbf{*.cpp files for source code}
\begin{itemize}
  \itemsep0em 
  \item Also called "Implementation File"
  \item Function implementations (can be in .h files as well)
  \item Source of compilation - aka "Translation Unit"
\end{itemize}
\textbf{*.h files for interfaces and templates}
\begin{itemize}
  \item Called "Header File"
  \item Declarations and definitions to be used in other implementation files.
  \item Textual inclusion through a pre-processor (C++20 will incorporate a "Module" mechanism)
  \item \#include "header.h"
\end{itemize}

\textbf{3 Phases of Compilation}
\begin{itemize}
  \itemsep0em 
  \item \textbf{Preprocessor} – Textual replacement of preprocessor directives (\#include)
  \item \textbf{Compiler} – Translation of C++ code into machine code (source file to object file)
  \item \textbf{Linker} - Combination of object files and libraries into libraries and executables
\end{itemize}

\subsection{Declarations and Definitions}
\textit{All things with a name that you use in a C++ program must be declared before you can do so!} \\
\textbf{Declaring Functions} \\
$<return-type> <function-name> (<parameters>);$







